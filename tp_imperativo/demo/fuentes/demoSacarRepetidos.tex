\small Las implicaciones mas importantes estan en \textcolor{NavyBlue}{AZUL}\\
\small Referirse a los estados mas arriba cuando sea necesario\\
\vspace{2mm}

\subsection{Pc $ \Rightarrow $ I}
\begin{itemize}
	\item $ ^1 $ \textcolor{NavyBlue}{longitud == $ |dnis| $}
	\item $ ^2 $ i == 0 $ \Rightarrow $ i $ \leq $ longitud
	\item $ ^2 $ i == 0  $ \Rightarrow $ 0 $ \leq $ i
	\item \textcolor{NavyBlue}{0 $ \leq $ i $ \leq $ longitud}
	\item $ ^3 $ i == 0 $ \Rightarrow $ \textcolor{NavyBlue}{result == [] } == [$ dnis_k $ $ | $ k$ \selec $[0..0), ($ \lnot $$ \exists $j$ \selec $[0..k)) $ dnis_j $ == $ dnis_k $]
	\item \textcolor{NavyBlue}{0 $ \leq $ i $ \leq $ longitud $ \land $ longitud == $ |dnis| $ $ \land $ result == [$ dnis_k $ $ | $ k$ \selec $[0..i), ($ \lnot $$ \exists $j$ \selec $[0..k)) $ dnis_j $ == $ dnis_k $ ]} \checkmark
\end{itemize}
\vspace{3mm}


\noindent $ ^1 $ Por poscondición del problema \emph{longitud} definido para el tipo Lista.\\
$ ^2 $ Porque longitud $\geq$ 0 (por ser $longitud == |dnis|$) y por propiedades de los números enteros\\
$ ^3 $ Propiedades de listas\\

\subsection{(I $ \land $ $ \lnot $B) $ \Rightarrow $ Qc}
\begin{itemize}
	\item $ ^1 $ (I $ \land $ $ \lnot $B) $ \Rightarrow $ i == longitud
	\item $ ^2 $ \textcolor{NavyBlue}{ result == [$ dnis_k $$ | $ k$ \selec $[0..longitud), ($ \lnot $$ \exists $j$ \selec $[0..k)) $ dnis_j $ == $ dnis_k $]} \checkmark
\end{itemize}

\vspace{3mm}

\noindent $ ^1 $ Propiedades de los números enteros\\
$ ^2 $ Se infiere de I\\

\subsection{Funcion variante decrece}
\begin{itemize}
	\item Fv: longitud - i
	\item $ i@e3 == i@e2 + 1 == i@e1 + 1 $
	\item $ ^1 $ $ i@e3 > i@e1 $
	\item $ ^1 $ $ - i@e3 < - i@e1 $
	\item $ ^2 $ \textcolor{NavyBlue}{ $ longitud - i@e3 < longitud - i@e1 $} \checkmark	
\end{itemize}
\vspace{3mm}

\noindent $ ^1 $ Propiedades de los números enteros\\
$ ^2 $ Sumando longitud a ambos lados\\

\subsection{I $ \land $ $ Fv \leq Cota $ $ \Rightarrow $ $ \lnot $ B}
\begin{itemize}
	\item Fv: longitud - i
	\item Cota: 0
	\item Fv $ \leq $ Cota $ \Rightarrow $ $ longitud - i \leq 0 $ 
	\item longitud $ \leq $ i
	\item B: i $ < $ longitud $ \Rightarrow $ $ \lnot $ B: longitud $ \leq $ i
	\item \textcolor{NavyBlue}{ $ \lnot $ B $ \Leftrightarrow $ Fv  $ \leq $ Cota} \checkmark	
\end{itemize}
\vspace{3mm}

\subsection{El cuerpo del ciclo preserva el invariante}
\begin{itemize}
	\item $ ^1 $ $ ^2 $ 0 $ \leq $ i@e1 $ \Rightarrow $ 0 $ \leq $ i@e1 + 1
	\item $ ^3 $ 0 $ \leq $ i@e3
	\item $ ^1 $ $^2 $ i@e1 $ < $ longitud $ \Rightarrow $ i@e1 + 1 $ < $ longitud + 1
	\item $ ^3 $ i@e3 $ \leq $ longitud	
	\item \textcolor{NavyBlue}{0 $ \leq $ i@e3 $ \leq $ longitud} \checkmark
	\item Si vale ( ($ dnis_{i@e1} $ $ \in $ result) $ \land $ (result == result@e1) ) entonces
	\item $ ^4 $ result == [$ dnis_k $$|$ k$ \selec $[0..i@e1 + 1), ($ \lnot $$ \exists $j$ \selec $[0..k)) $ dnis_j $ == $ dnis_k $]
	\item $ ^3 $ \textcolor{NavyBlue}{result == [$ dnis_k $$|$ k$ \selec $[0..i@e3), ($ \lnot $$ \exists $j$ \selec $[0..k)) $ dnis_j $ == $ dnis_k $]} \checkmark
	\item Si vale ( ($ dnis_{i@e1} $ $ \notin $ result) $ \land $ (result == result@e1 ++ [$ dnis_i@e1 $]) ) entonces
	\item result@e1 == [$ dnis_k $$|$ k$ \selec $[0..i@e1), ($ \lnot $$ \exists $j$ \selec $[0..k)) $ dnis_j $ == $ dnis_k $]
	\item result@e1 ++ [$ dnis_i@e1 $] == [$ dnis_k $| k$ \selec $[0..i@e1), ($ \lnot $$ \exists $j$ \selec $[0..k)) $ dnis_j $ == $ dnis_k $] ++ [$ dnis_i@e1 $]
	\item $ ^5 $ result@e1 ++ [$ dnis_i@e1 $] == [$ dnis_k $ $|$ k$ \selec $[0..i@e1], ($ \lnot $$ \exists $j$ \selec $[0..k)) $ dnis_j $ == $ dnis_k $]
	\item $ ^6 $ $ ^7 $ result@e3 == [$ dnis_k $$|$ k$ \selec $[0..i@e1 + 1), ($ \lnot $$ \exists $j$ \selec $[0..k)) $ dnis_j $ == $ dnis_k $]
	\item $ ^3 $ \textcolor{NavyBlue}{result@e3 == [$ dnis_k $$|$ k$ \selec $[0..i@e3), ($ \lnot $$ \exists $j$ \selec $[0..k)) $ dnis_j $ == $ dnis_k $]} \checkmark
\end{itemize}
\vspace{3mm}

\noindent $ ^1 $ Propiedades de los números enteros\\
$ ^2 $ Vale I $ \land $ B en e1\\
$ ^3 $ i@e3 == i@e1 + 1\\
$ ^4 $ Por propiedad de listas como la condicion de la lista no se cumple, podemos utilizar i@e1 + 1\\
$ ^5 $ Por propiedad de listas podemos incluir al elemento concatenado como parte de la lista por comprension\\
$ ^6 $ Por propiedad de intervalos\\
$ ^7 $ result@e3 == result@e2 == result@e1 ++ [$ dnis_i@e1 $]\\
